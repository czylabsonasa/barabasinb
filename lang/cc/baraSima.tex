%\documentclass[14pt]{article}
\documentclass[14pt]{extarticle}

\usepackage[margin=1.5cm,nohead]{geometry}
\usepackage[utf8]{inputenc}
\usepackage[hungarian]{babel}
\usepackage[%hypertex,
                 unicode=true,
                 plainpages = false, 
                 pdfpagelabels, 
                 bookmarks=true,
                 bookmarksnumbered=true,
                 bookmarksopen=true,
                 breaklinks=true,
                 backref=false,
                 colorlinks=true,
                 linkcolor = blue,
                 urlcolor  = blue,
                 citecolor = red,
                 anchorcolor = green,
                 hyperindex = true,
                 hyperfigures
]{hyperref}
\hypersetup{
 pdftitle={Barabasi},
 pdfauthor={Czylabson Asa},
 pdfsubject={Hold Föld Nap}
}
\usepackage{amsmath}
\usepackage{amsfonts}
\usepackage{amssymb}
\usepackage{graphicx}
\usepackage{type1cm}

\usepackage{setspace}
\onehalfspacing

\newcommand{\zjel}[1]{%
{ \left( #1 \right) }
}

\begin{document}
\begin{spacing}{1.5}

	\par\centerline{Barabási dallam}

	\begin{itemize}
		\item néha az $=$ helyett $\approx$ értendő

		\item $t=0$ a kezdő pillanat, ekkor az összfokszám: $d_{0}$. A csucsok száma
		$v_{0}\ge m.$ 
		\begin{itemize}
			\item a kezdő lehet pl. egy teljes $m+1$ gráf, ekkor $d_{0}=(m+1)m$, 
			\newline ezzel kényelmes "dolgozni", mert az első lépés megtehető és 
			nem determinisztikus.
		\end{itemize}

		\item $d_{t}$ az összfokszám a $t$ pillanatban
		\begin{itemize}
			\item minden lépésben létrejön $m$ új él, ami $2m$ fokszámnövekedés:
			\newline $d_{t}=d_{0}+t2m$. 
		\end{itemize}
		\item $d_{t}(d)$ a $t$ pillanatban a $d$ fokúak száma.
		\item először $d_{t}(m)$-et nézzük:
		\begin{itemize}
			\item $d_{0}(m),d_{0}(m-1)$ adott.

			\item $d_{t+1}(m)=d_{t}(m)+1+X-Y$, ahol $X$ a keletkező, $Y$ a megszűnő $m$ fokúak véletlentől függő száma.

%bin 
			\item {\it Hanyagoljuk} el egyrészt a régi csúcsok választási lépéseinek, másrészt az $X,Y$ párnak a függőségét:
			\newline $X$-el $Bin\zjel{m,\frac{(m-1)d_{t}(m-1)}{d_{t}}}$-ként, 
         \newline $Y$-al $Bin\zjel{m,\frac{md_{t}(m)}{d_{t}}}$-ként dolgozunk tovább.


%atlagok
			\item {\it Átlagokra} áttérve (ide kellene egy feltételes várható értékes kitérő):
			\newline $\overline{{d}_{t+1}(m)}=\overline{{d}_{t}(m)}+1+
         \frac{m(m-1)\overline{{d}_{t}(m-1)}}{d_{t}}-\frac{m^{2}\overline{{d}_{t}(m)}}{d_{t}}$
         \newline a $\frac{(m-1)\overline{{d}_{t}(m-1)}}{d_{t}}$ tagot elhanyagoljuk, mivel az $m-1$ 
         fokú csúcsok száma nem nő és $d_{{t}}\uparrow\infty:$
         \newline $\overline{{d}_{t+1}(m)}=\overline{{d}_{t}(m)}+1-\frac{m\overline{{d}_{t}(m)}}{d_{t}}$
%egyensuly
			\item Feltételezzük, hogy a gráf közel {\it egyensúlyi helyzet}be "kerül" nagy $t$-re, és 
			\newline ezt $\overline{d_{t}(m)}=c_{m}d_{t}$-vel fejezzük ki. Ekkor
			\newline $c_{m}2m=1-m^{2}c_{m}$, azaz  
			$c_{m}=\frac{1}{m(m+2)}.$

		\end{itemize}
		\item $d_{t}(d)$ $d>m$-re:
		\begin{itemize}
			\item $d_{t+1}(d)=d_{t}(d)+X-Y$, ahol $X$ az éppen keletkező $d$-fokúakat, 
			$Y$ a megszűnőket számolja.
			\newline Itt is $X$-et $Bin\zjel{m,\frac{(d-1)d_{t}(d-1)}{d_{t}}}$-nel 
         \newline illetve $Y$-t $Bin\zjel{m,\frac{dd_{t}(d)}{d_{t}}}$-nel helyettesítjük,
         \newline átlagolunk, és reménykedünk az $\overline{d_{t}(d)}=c_{d}d_{t}$ 
         "egyensúlyban":
         \newline
			$\overline{{d}_{t+1}(d)}=\overline{{d}_{t}(d)}+
         \frac{m(d-1)\overline{{d}_{t}(d-1)}}{d_{t}}-\frac{md\overline{{d}_{t}(d)}}{d_{t}}$			
         \newline $c_{d}2m=m(d-1)c_{d-1}-mdc_{d}$, azaz 
         $c_{d}=\frac{d-1}{d+2}c_{d-1}.$
		\end{itemize}
	\end{itemize}


\end{spacing}
\end{document}
